% Options for packages loaded elsewhere
\PassOptionsToPackage{unicode}{hyperref}
\PassOptionsToPackage{hyphens}{url}
\PassOptionsToPackage{dvipsnames,svgnames,x11names}{xcolor}
%
\documentclass[
  number]{elsarticle}

\usepackage{amsmath,amssymb}
\usepackage{iftex}
\ifPDFTeX
  \usepackage[T1]{fontenc}
  \usepackage[utf8]{inputenc}
  \usepackage{textcomp} % provide euro and other symbols
\else % if luatex or xetex
  \usepackage{unicode-math}
  \defaultfontfeatures{Scale=MatchLowercase}
  \defaultfontfeatures[\rmfamily]{Ligatures=TeX,Scale=1}
\fi
\usepackage{lmodern}
\ifPDFTeX\else  
    % xetex/luatex font selection
\fi
% Use upquote if available, for straight quotes in verbatim environments
\IfFileExists{upquote.sty}{\usepackage{upquote}}{}
\IfFileExists{microtype.sty}{% use microtype if available
  \usepackage[]{microtype}
  \UseMicrotypeSet[protrusion]{basicmath} % disable protrusion for tt fonts
}{}
\makeatletter
\@ifundefined{KOMAClassName}{% if non-KOMA class
  \IfFileExists{parskip.sty}{%
    \usepackage{parskip}
  }{% else
    \setlength{\parindent}{0pt}
    \setlength{\parskip}{6pt plus 2pt minus 1pt}}
}{% if KOMA class
  \KOMAoptions{parskip=half}}
\makeatother
\usepackage{xcolor}
\setlength{\emergencystretch}{3em} % prevent overfull lines
\setcounter{secnumdepth}{5}
% Make \paragraph and \subparagraph free-standing
\makeatletter
\ifx\paragraph\undefined\else
  \let\oldparagraph\paragraph
  \renewcommand{\paragraph}{
    \@ifstar
      \xxxParagraphStar
      \xxxParagraphNoStar
  }
  \newcommand{\xxxParagraphStar}[1]{\oldparagraph*{#1}\mbox{}}
  \newcommand{\xxxParagraphNoStar}[1]{\oldparagraph{#1}\mbox{}}
\fi
\ifx\subparagraph\undefined\else
  \let\oldsubparagraph\subparagraph
  \renewcommand{\subparagraph}{
    \@ifstar
      \xxxSubParagraphStar
      \xxxSubParagraphNoStar
  }
  \newcommand{\xxxSubParagraphStar}[1]{\oldsubparagraph*{#1}\mbox{}}
  \newcommand{\xxxSubParagraphNoStar}[1]{\oldsubparagraph{#1}\mbox{}}
\fi
\makeatother


\providecommand{\tightlist}{%
  \setlength{\itemsep}{0pt}\setlength{\parskip}{0pt}}\usepackage{longtable,booktabs,array}
\usepackage{calc} % for calculating minipage widths
% Correct order of tables after \paragraph or \subparagraph
\usepackage{etoolbox}
\makeatletter
\patchcmd\longtable{\par}{\if@noskipsec\mbox{}\fi\par}{}{}
\makeatother
% Allow footnotes in longtable head/foot
\IfFileExists{footnotehyper.sty}{\usepackage{footnotehyper}}{\usepackage{footnote}}
\makesavenoteenv{longtable}
\usepackage{graphicx}
\makeatletter
\def\maxwidth{\ifdim\Gin@nat@width>\linewidth\linewidth\else\Gin@nat@width\fi}
\def\maxheight{\ifdim\Gin@nat@height>\textheight\textheight\else\Gin@nat@height\fi}
\makeatother
% Scale images if necessary, so that they will not overflow the page
% margins by default, and it is still possible to overwrite the defaults
% using explicit options in \includegraphics[width, height, ...]{}
\setkeys{Gin}{width=\maxwidth,height=\maxheight,keepaspectratio}
% Set default figure placement to htbp
\makeatletter
\def\fps@figure{htbp}
\makeatother

\makeatletter
\@ifpackageloaded{caption}{}{\usepackage{caption}}
\AtBeginDocument{%
\ifdefined\contentsname
  \renewcommand*\contentsname{Table of contents}
\else
  \newcommand\contentsname{Table of contents}
\fi
\ifdefined\listfigurename
  \renewcommand*\listfigurename{List of Figures}
\else
  \newcommand\listfigurename{List of Figures}
\fi
\ifdefined\listtablename
  \renewcommand*\listtablename{List of Tables}
\else
  \newcommand\listtablename{List of Tables}
\fi
\ifdefined\figurename
  \renewcommand*\figurename{Figure}
\else
  \newcommand\figurename{Figure}
\fi
\ifdefined\tablename
  \renewcommand*\tablename{Table}
\else
  \newcommand\tablename{Table}
\fi
}
\@ifpackageloaded{float}{}{\usepackage{float}}
\floatstyle{ruled}
\@ifundefined{c@chapter}{\newfloat{codelisting}{h}{lop}}{\newfloat{codelisting}{h}{lop}[chapter]}
\floatname{codelisting}{Listing}
\newcommand*\listoflistings{\listof{codelisting}{List of Listings}}
\makeatother
\makeatletter
\makeatother
\makeatletter
\@ifpackageloaded{caption}{}{\usepackage{caption}}
\@ifpackageloaded{subcaption}{}{\usepackage{subcaption}}
\makeatother
\ifLuaTeX
  \usepackage{selnolig}  % disable illegal ligatures
\fi
\usepackage[]{natbib}
\bibliographystyle{elsarticle-num}
\usepackage{bookmark}

\IfFileExists{xurl.sty}{\usepackage{xurl}}{} % add URL line breaks if available
\urlstyle{same} % disable monospaced font for URLs
\hypersetup{
  pdftitle={Mapping landscape suitability for forest thinning to reduce evapotranspiration and enhance groundwater recharge in Arizona},
  pdfauthor={Ryan E Lima; Neha Gupta; Travis Zalesky; Temuulen Tsagaan Sankey; Abraham E Springer; Katherine Jacobs},
  pdfkeywords={suitability mapping, Forest thinning, Water
yield, groundwater recharge, GIS-MCDM, AHP},
  colorlinks=true,
  linkcolor={blue},
  filecolor={Maroon},
  citecolor={Blue},
  urlcolor={Blue},
  pdfcreator={LaTeX via pandoc}}

\setlength{\parindent}{6pt}
\begin{document}

\begin{frontmatter}
\title{Mapping landscape suitability for forest thinning to reduce
evapotranspiration and enhance groundwater recharge in Arizona}
\author[1]{Ryan E Lima%
\corref{cor1}%
}
 \ead{ryan.lima@nau.edu} 
\author[2]{Neha Gupta%
%
}

\author[2]{Travis Zalesky%
%
}

\author[1]{Temuulen Tsagaan Sankey%
%
}

\author[1]{Abraham E Springer%
%
}

\author[2]{Katherine Jacobs%
%
}


\affiliation[1]{organization={Northern Arizona
University},,postcodesep={}}
\affiliation[2]{organization={University of Arizona},,postcodesep={}}

\cortext[cor1]{Corresponding author}






        
\begin{abstract}
Literature on the relationship between forest thinning and water yield
was used to develop suitability criteria to map where forest treatment
is most likely to enhance groundwater recharge across the Coconino
National Forest in Arizona. Rechage in the region is ephemeral and
focused in periods of snowmelt and locations of enhanced permeability
when soil moisture exceeds threshold levels. Our approach combines
thematic maps of criteria such as average precipitation, snow dominance,
slope, aspect,landscape morphology, forest basal area, canopy cover,
lithology and hydrologic soil type into a GIS-Multi-Criteria Decision
Making model. Pairwise comparisons were made between criteria, and
Analytic Hierachy Process was used as a weighting method.
\end{abstract}





\begin{keyword}
    suitability mapping \sep Forest thinning \sep Water
yield \sep groundwater recharge \sep GIS-MCDM \sep 
    AHP
\end{keyword}
\end{frontmatter}
    
\section{Introduction}\label{introduction}

Warming associated with anthropocentric climate change has increased the
number of extreme hydroclimate events in the Colorado River Basin.
Droughts, heatwaves, and floods have more than doubled in frequency
since 2010 \citep{bennett_concurrent_2021}.Since 2000, the Colorado
River Basin has been in the midst of a historic drought
\citep{meko_treering_2022, williams_rapid_2022}. During that time,
streamflow in the Colorado River has declined by 19\% relative to the
1906-1999 average
\citep{hogan_recent_2024, udall_twentyfirst_2017}.Rapid population
growth in the Southwest and Arizona, in particular, is increasing the
demands on already strained water supplies in the State. Reductions in
streamflow have increased reliance on groundwater pumping, while
groundwater levels have declined for decades in many groundwater basins
across Arizona \citep{tadych_historical_2024}.

Concurrently, the risk of catastrophic wildfires is increasing in
Western forests--an emerging driver of runoff change that will increase
the impact on the water supply \citep{williams_rapid_2022}. Forest
structure in Northern Arizona and New Mexico has changed significantly
post-Euro-American settlement. Many forests are overstocked relative to
pre-settlement conditions due to grazing, logging, and wildfire
exclusion \citep{covington_southwestern_1994, friederici2013}. These
changes have increased the risk of catastrophic wildfire
\citep{allen_ecological_2002}. Rising temperatures and related droughts
have contributed to extensive tree mortality from wildfire, disease, and
insect infestation \citep{berner_tree_2017}. Warming temperatures have
tripled the frequency and quadrupled the size of wildfires since 2000
{[}\citep{iglesias2022}{]}.

Landscape-scale forest restoration efforts have been planned or
implemented across much of Arizona. For example, the Four Forest
Restoration Initiative (4FRI) includes plans for restoration across over
1 million hectares of Arizona's forests
\citep{schultz_collaborative_2012}. The primary goal of restoration
efforts is to reduce wildfire risk
\citep{allen_ecological_2002, friederici2013}. However, numerous studies
have linked forest treatments to increased water yields in semi-arid
forests and have emphasized the role of forest restoration in improving
hydrologic services and increasing water availability
\citep{bosch_review_1982, baker_effects_1986, gottfried_moderate_1991, smerdon_overview_2009, zou_streamflow_2010, wyatt_estimating_2013, moreno_modeling_2015, simonit_impact_2015, wyatt_semiarid_2015, odonnell_forest_2018, schenk_impacts_2020, hibbert1979}.
Forest treatments such as thinning and burning can significantly impact
the hydrologic cycle of forests \citep{del_campo_global_2022}. For
example, forest thinning in Arizona has been associated with increased
snow cover days
\citep{sankey_multi-scale_2015, belmonte_uav-based_2021, donager_integrating_2021},
greater soil moisture \citep{belmonte_soil_2022, sankey_thinning_2022},
and greater forest canopy moisture \citep{sankey_regionalscale_2021}.

While the connection between forest treatment and water yield is well
documented, the response of forests to treatments is complex and
non-linear and differs across forest types, with treatment level, and
along aspect and elevational gradients
\citep{del_campo_global_2022, biederman_streamflow_2022, zou_streamflow_2010, hibbert1979, moore_physical_2005}.
Regardless of the potential for increased water yield, the enhancement
of groundwater recharge rarely, if ever, ranks among the primary
motivations for forest treatment even among projects with the stated
goal of improving watershed health
\citep{stanturf2014, filoso2017, allen_ecological_2002, friederici2013, odonnell2016}.
This study examines forest restoration through the lens of groundwater
recharge enhancement and aimes to identify potential recharge zones. We
map suitability for forest thinning with the goal of enhancing recharge.
Suitability maps like these may complement (or supplement) existing
frameworks for prioritizing landscape-scale forest management.

Suitability mapping, and particularly GIS-based Multi-criteria decision
making (GIS-MCDM), is widely used to map potential recharge zones and
areas suitable for Managed Aquifer Recharge (MAR), but to our knowledge,
it has not yet been implemented to map incidental recharge, or recharge
enhancement potential, from forest thinning
\citep{fathi2021, rajashekar2023}. Pairwise comparisons were made
between criteria including forest basal area, canopy cover, average
precipitation, snow dominance, slope, aspect, landscape morphology,
forest density, lithology, and hydrologic soil type, and the Analytic
Hierarchy Process (AHP) was used as a weighting method.

\begin{center}\rule{0.5\linewidth}{0.5pt}\end{center}

\subsection{Literature Reviewed}\label{literature-reviewed}

\begin{itemize}
\item
  How forest thinning may enhance recharge

  \begin{itemize}
  \item
    Reduced canopy interception, increased snow retention
  \item
    decreases ET - TEKI fort valley research
  \item
    Increases Soil Moisture
  \item
    Precipitation thresholds in thinning literature
  \end{itemize}
\item
  Caveats

  \begin{itemize}
  \item
    Low Elevations - less water yield in low elevation forests in the
    Salt-Verde System \citep{biederman_streamflow_2022}
  \item
    Sunny Aspects - increased sublimation with decreased canopy cover
    \citep{biederman_recent_2015}
  \end{itemize}
\end{itemize}

\section{Methods}\label{methods}

\subsection{Study Area}\label{study-area}

The Coconino National Forest is located in Northern Arizona near the
cities of Flagstaff and Sedona. It ranges in elevation from 790 \(m\)
near the Verde River to 3,851 \(m\) at the summit of Humphreys Peak. It
is located within the largest contiguous ponderosa pine (\emph{Pinus
ponderosa}) forest in North America. Ponderosa pine covers roughly 40\%
of the national forest an area of about 340,000 hectares, primarily
between 2000 - 2400 \(m\) in elevation.

\begin{itemize}
\item
  The Coconino National Forest also spans the Mogollon Rim, which has
  been identified as a an important recharge area for the regional
  groundwater aquifers
\item
  Of the estimated 1,740,000 acre-feet of precipitation that falls on
  the Mogollon Rim annually, about 8 percent is estimated to recharge
  the regional aquifers \citep{parker2005}.
\end{itemize}

\subsection{Initial Suitability
Screen}\label{initial-suitability-screen}

Areas were considered unsuitable for the following reasons:

\begin{enumerate}
\def\labelenumi{\arabic{enumi}.}
\item
  Areas with Max annual precipitation 1991 - 2020 \textless{} 500mm
  {[}source: AORC Max Annual Precipitation 1991 - 2020{]}
\item
  All Landcover types other than 410 Deciduous Forest,42 - Evergreen
  Forest, \& 43 - Mixed Forest {[}source: NLCD Landcover 2021{]}
\item
  Forest Vegetation types that include attributes EVT\_LF =
  {[}``Water'', ``Shrub'', ``Sparse'', ``Herb'', ``Developed'',
  ``Barren'', ``Agriculture''{]} and EVT\_PHYS = {[}``Developed'',
  ``Riparian'', ``Agriculture''{]} and including keywords in EVT\_NAME =
  {[}``Madrean'', ``Savanna''{]}. {[}Source: LF2023\_240\_EVT{]}
  Landfire
\item
  Wilderness Areas
\end{enumerate}

\subsection{Suitability Criteria}\label{suitability-criteria}

\subsubsection{Topographic Relative Moisture
Index}\label{topographic-relative-moisture-index}

Topographic Relative Moisture Index (TRMI) incorporates several
topographic parameters that influence moisture dynamics including slope
gradient, aspect, relative elevation (or topographic position) and
landscape convexity or concavity \citep{parker1982}. TRMI data comes
from the Southwest Regional Gap Analysis
\href{https://swregap.org/data/}{(SWReGAP)} where terrain features are
separated into 10 categories ranked in terms of soil moisture. Del Campo
and others \citep{del_campo_effectiveness_2019} examined below ground
hydrological processes in thinned semi-arid watersheds and found that
sites with high antecedent soil moisture had the highest response, with
drainage to deeper soil layers increasing by 50mm/year relative to
control sites.

\begin{longtable}[]{@{}
  >{\raggedright\arraybackslash}p{(\columnwidth - 4\tabcolsep) * \real{0.2361}}
  >{\raggedright\arraybackslash}p{(\columnwidth - 4\tabcolsep) * \real{0.5278}}
  >{\raggedright\arraybackslash}p{(\columnwidth - 4\tabcolsep) * \real{0.2361}}@{}}
\toprule\noalign{}
\begin{minipage}[b]{\linewidth}\raggedright
Class Name
\end{minipage} & \begin{minipage}[b]{\linewidth}\raggedright
Description
\end{minipage} & \begin{minipage}[b]{\linewidth}\raggedright
Suitability Value
\end{minipage} \\
\midrule\noalign{}
\endhead
\bottomrule\noalign{}
\endlastfoot
Valley Flats & These areas are typically the lowest points in a
landscape where water naturally accumulates, making them the moistest. &
10 \\
Very Moist Steep Slopes & Despite the slope, these areas (often with
north-facing or shaded aspects in the Northern Hemisphere) retain
moisture due to cool temperatures and reduced evaporation. & 9 \\
Toe Slopes & These are the lower parts of slopes where water tends to
collect after flowing downhill, making them moist but slightly less so
than valley flats. & 8 \\
Cool Aspect Scarps, Cliffs, Canyons & North-facing scarps and shaded
canyons are cooler and retain moisture better than exposed slopes,
particularly in arid regions. & 7 \\
Nearly Level Plateaus or Terraces & These flat areas may retain moderate
amounts of moisture but are generally more exposed to sunlight and wind
than valleys or steep moist slopes. & 6 \\
Gently Sloping Ridges & These ridges are not as steep, so water may
infiltrate rather than run off completely, but they are still relatively
dry due to elevation. & 5 \\
Moderately Moist Steep Slopes & These are typically slopes with
intermediate aspects or conditions, retaining some moisture but less
than moist slopes. & 4 \\
Moderately Dry Slopes & These areas have greater runoff and less water
retention, often due to steeper gradients and/or sun exposure. & 3 \\
Very Dry Steep Slopes & Steep slopes with sun exposure (e.g.,
south-facing in the Northern Hemisphere) promote high runoff and
evaporation, leaving them very dry. & 2 \\
Hot Aspect Scarps, Cliffs, and Canyons & South-facing cliffs and canyons
with maximum sun exposure (in the Northern Hemisphere) are the driest
due to high evaporation and limited water retention. & 1 \\
\end{longtable}

\subsubsection{Basal Area}\label{basal-area}

We extracted basal area estimates from the TreeMap 2016
\citep{riley2022} conus dataset.

\subsubsection{Snowfall Dominance}\label{snowfall-dominance}

Ask Patrick Broxton how it was calculated.

\subsubsection{Mean Annual
Precipitation}\label{mean-annual-precipitation}

Utilized AORC Retrospective forcing data for average water year (WY)
precipitation 1991 - 2020

\subsubsection{Subsurface Infiltration
Capacity}\label{subsurface-infiltration-capacity}

We utilized the Global Hydrological Maps of Permeability and Porosity
(GLHYMPS) to estimate subsurface infiltration capacity
\citep{gleeson2014}

\subsubsection{Soil Hydrologic Type}\label{soil-hydrologic-type}

Data from gNATSGO was used to estimate the soil's ability to infiltrate
water, we used the Hydrologic Soil Type.

\subsubsection{Canopy Cover}\label{canopy-cover}

Canopy Cover was obtained from the 2021 National Land Cover Dataset ,
which provides an estimate of Total Canopy Cover

\subsection{Weighting}\label{weighting}

\begin{longtable}[]{@{}ll@{}}
\toprule\noalign{}
Criteria & Weight (\%) \\
\midrule\noalign{}
\endhead
\bottomrule\noalign{}
\endlastfoot
Subsurface Infiltration & 10.12 \\
Canopy Cover & 9.82 \\
Basal Area & 15.06 \\
Topographic Moisture Index & 29.48 \\
Mean annual Precipitation & 10.73 \\
Snowfall Fraction & 14.77 \\
\end{longtable}

\section{Results}\label{results}

We identified 30 forest patches of between 50 and 60 Hectares that we
consider suitable for thinning for the purposes of enhancing groundwater
recharge. The patches have mean suitability values ranging from 5.8 to
6.25

\begin{figure}[H]

{\centering \includegraphics{images/Thinning_Suitability_Map.jpg}

}

\caption{Figure x: Suitability Map for Thinning to Enhance Groundwater
Recharge}

\end{figure}%

\section{Discussion}\label{discussion}

\subsection{Acknowledgments}\label{acknowledgments}

Phasellus interdum tincidunt ex, a euismod massa pulvinar at. Ut
fringilla ut nisi nec volutpat. Morbi imperdiet congue tincidunt.
Vivamus eget rutrum purus. Etiam et pretium justo. Donec et egestas sem.
Donec molestie ex sit amet viverra egestas. Nullam justo nulla,
fringilla at iaculis in, posuere non mauris. Ut eget imperdiet elit.

\subsection{Open research}\label{open-research}

Phasellus interdum tincidunt ex, a euismod massa pulvinar at. Ut
fringilla ut nisi nec volutpat. Morbi imperdiet congue tincidunt.
Vivamus eget rutrum purus. Etiam et pretium justo. Donec et egestas sem.
Donec molestie ex sit amet viverra egestas. Nullam justo nulla,
fringilla at iaculis in, posuere non mauris. Ut eget imperdiet elit.

\subsection*{References}\label{references}
\addcontentsline{toc}{subsection}{References}

\renewcommand{\bibsection}{}
\bibliography{bibliography.bib}




\end{document}
