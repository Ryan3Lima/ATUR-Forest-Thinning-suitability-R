% Options for packages loaded elsewhere
\PassOptionsToPackage{unicode}{hyperref}
\PassOptionsToPackage{hyphens}{url}
\PassOptionsToPackage{dvipsnames,svgnames,x11names}{xcolor}
%
\documentclass[
  number]{elsarticle}

\usepackage{amsmath,amssymb}
\usepackage{iftex}
\ifPDFTeX
  \usepackage[T1]{fontenc}
  \usepackage[utf8]{inputenc}
  \usepackage{textcomp} % provide euro and other symbols
\else % if luatex or xetex
  \usepackage{unicode-math}
  \defaultfontfeatures{Scale=MatchLowercase}
  \defaultfontfeatures[\rmfamily]{Ligatures=TeX,Scale=1}
\fi
\usepackage{lmodern}
\ifPDFTeX\else  
    % xetex/luatex font selection
\fi
% Use upquote if available, for straight quotes in verbatim environments
\IfFileExists{upquote.sty}{\usepackage{upquote}}{}
\IfFileExists{microtype.sty}{% use microtype if available
  \usepackage[]{microtype}
  \UseMicrotypeSet[protrusion]{basicmath} % disable protrusion for tt fonts
}{}
\makeatletter
\@ifundefined{KOMAClassName}{% if non-KOMA class
  \IfFileExists{parskip.sty}{%
    \usepackage{parskip}
  }{% else
    \setlength{\parindent}{0pt}
    \setlength{\parskip}{6pt plus 2pt minus 1pt}}
}{% if KOMA class
  \KOMAoptions{parskip=half}}
\makeatother
\usepackage{xcolor}
\setlength{\emergencystretch}{3em} % prevent overfull lines
\setcounter{secnumdepth}{5}
% Make \paragraph and \subparagraph free-standing
\makeatletter
\ifx\paragraph\undefined\else
  \let\oldparagraph\paragraph
  \renewcommand{\paragraph}{
    \@ifstar
      \xxxParagraphStar
      \xxxParagraphNoStar
  }
  \newcommand{\xxxParagraphStar}[1]{\oldparagraph*{#1}\mbox{}}
  \newcommand{\xxxParagraphNoStar}[1]{\oldparagraph{#1}\mbox{}}
\fi
\ifx\subparagraph\undefined\else
  \let\oldsubparagraph\subparagraph
  \renewcommand{\subparagraph}{
    \@ifstar
      \xxxSubParagraphStar
      \xxxSubParagraphNoStar
  }
  \newcommand{\xxxSubParagraphStar}[1]{\oldsubparagraph*{#1}\mbox{}}
  \newcommand{\xxxSubParagraphNoStar}[1]{\oldsubparagraph{#1}\mbox{}}
\fi
\makeatother


\providecommand{\tightlist}{%
  \setlength{\itemsep}{0pt}\setlength{\parskip}{0pt}}\usepackage{longtable,booktabs,array}
\usepackage{calc} % for calculating minipage widths
% Correct order of tables after \paragraph or \subparagraph
\usepackage{etoolbox}
\makeatletter
\patchcmd\longtable{\par}{\if@noskipsec\mbox{}\fi\par}{}{}
\makeatother
% Allow footnotes in longtable head/foot
\IfFileExists{footnotehyper.sty}{\usepackage{footnotehyper}}{\usepackage{footnote}}
\makesavenoteenv{longtable}
\usepackage{graphicx}
\makeatletter
\def\maxwidth{\ifdim\Gin@nat@width>\linewidth\linewidth\else\Gin@nat@width\fi}
\def\maxheight{\ifdim\Gin@nat@height>\textheight\textheight\else\Gin@nat@height\fi}
\makeatother
% Scale images if necessary, so that they will not overflow the page
% margins by default, and it is still possible to overwrite the defaults
% using explicit options in \includegraphics[width, height, ...]{}
\setkeys{Gin}{width=\maxwidth,height=\maxheight,keepaspectratio}
% Set default figure placement to htbp
\makeatletter
\def\fps@figure{htbp}
\makeatother

\makeatletter
\@ifpackageloaded{caption}{}{\usepackage{caption}}
\AtBeginDocument{%
\ifdefined\contentsname
  \renewcommand*\contentsname{Table of contents}
\else
  \newcommand\contentsname{Table of contents}
\fi
\ifdefined\listfigurename
  \renewcommand*\listfigurename{List of Figures}
\else
  \newcommand\listfigurename{List of Figures}
\fi
\ifdefined\listtablename
  \renewcommand*\listtablename{List of Tables}
\else
  \newcommand\listtablename{List of Tables}
\fi
\ifdefined\figurename
  \renewcommand*\figurename{Figure}
\else
  \newcommand\figurename{Figure}
\fi
\ifdefined\tablename
  \renewcommand*\tablename{Table}
\else
  \newcommand\tablename{Table}
\fi
}
\@ifpackageloaded{float}{}{\usepackage{float}}
\floatstyle{ruled}
\@ifundefined{c@chapter}{\newfloat{codelisting}{h}{lop}}{\newfloat{codelisting}{h}{lop}[chapter]}
\floatname{codelisting}{Listing}
\newcommand*\listoflistings{\listof{codelisting}{List of Listings}}
\makeatother
\makeatletter
\makeatother
\makeatletter
\@ifpackageloaded{caption}{}{\usepackage{caption}}
\@ifpackageloaded{subcaption}{}{\usepackage{subcaption}}
\makeatother
\ifLuaTeX
  \usepackage{selnolig}  % disable illegal ligatures
\fi
\usepackage[]{natbib}
\bibliographystyle{elsarticle-num}
\usepackage{bookmark}

\IfFileExists{xurl.sty}{\usepackage{xurl}}{} % add URL line breaks if available
\urlstyle{same} % disable monospaced font for URLs
\hypersetup{
  pdftitle={Mapping landscape suitability for forest thinning to reduce evapotranspiration and enhance groundwater recharge in Arizona},
  pdfauthor={Ryan E Lima; Neha Gupta; Travis Zalesky; Temuulen Tsagaan Sankey; Abraham E Springer; Katherine Jacobs},
  pdfkeywords={suitability mapping, Forest thinning, Water
yield, groundwater recharge, GIS-MCDM, AHP},
  colorlinks=true,
  linkcolor={blue},
  filecolor={Maroon},
  citecolor={Blue},
  urlcolor={Blue},
  pdfcreator={LaTeX via pandoc}}

\setlength{\parindent}{6pt}
\begin{document}

\begin{frontmatter}
\title{Mapping landscape suitability for forest thinning to reduce
evapotranspiration and enhance groundwater recharge in Arizona}
\author[1]{Ryan E Lima%
\corref{cor1}%
}
 \ead{ryan.lima@nau.edu} 
\author[2]{Neha Gupta%
%
}

\author[2]{Travis Zalesky%
%
}

\author[1]{Temuulen Tsagaan Sankey%
%
}

\author[1]{Abraham E Springer%
%
}

\author[2]{Katherine Jacobs%
%
}


\affiliation[1]{organization={Northern Arizona
University},,postcodesep={}}
\affiliation[2]{organization={University of Arizona},,postcodesep={}}

\cortext[cor1]{Corresponding author}






        
\begin{abstract}
Literature on the relationship between forest thinning and water yield
was used to develop suitability criteria to map where forest treatment
is most likely to enhance groundwater recharge across the Coconino
National Forest in Arizona. Rechage in the region is ephemeral and
focused in periods of snowmelt and locations of enhanced permeability
when soil moisture exceeds threshold levels. Our approach combines
thematic maps of criteria such as average precipitation, snow dominance,
slope, aspect,landscape morphology, forest density, lithology and
hydrologic soil type into a GIS-Multi-Criteria Decision Making model.
Pairwise comparisons were made between criteria, and Analytic Hierachy
Process was used as a weighting method.
\end{abstract}





\begin{keyword}
    suitability mapping \sep Forest thinning \sep Water
yield \sep groundwater recharge \sep GIS-MCDM \sep 
    AHP
\end{keyword}
\end{frontmatter}
    
\section{Introduction}\label{introduction}

Warming associated with anthropocentric climate change has increased the
number of extreme hydroclimate events in the Colorado River Basin.
Droughts, heatwaves, and floods have more than doubled in frequency
since 2010 \citep{bennett_concurrent_2021}.Since 2000, the Colorado
River Basin has been in the midst of a historic drought
\citep{meko_treering_2022, williams_rapid_2022}. During that time,
streamflow in the Colorado River has declined by 19\% relative to the
1906-1999 average
\citep{hogan_recent_2024, udall_twentyfirst_2017}.Rapid population
growth in the Southwest and Arizona, in particular, is increasing the
demands on already strained water supplies in the State. Reductions in
streamflow have increased reliance on groundwater pumping, while
groundwater levels have declined for decades in many groundwater basins
across Arizona \citep{tadych_historical_2024}.

Concurrently, the risk of catastrophic wildfires is increasing in
Western forests--an emerging driver of runoff change that will increase
the impact on the water supply \citep{williams_rapid_2022}. Forest
structure in the forests of Northern Arizona and New Mexico has changed
significantly post-Euro-American settlement. Many forests are
overstocked relative to pre-settlement conditions due to grazing,
logging, and wildfire exclusion
\citep{covington_southwestern_1994, friederici2013}. These changes have
increasing the risk of catastrophic wildfire
\citep{allen_ecological_2002}. Rising temperatures and related droughts
have contributed to extensive tree mortality from wildfire, disease, and
insect infestation \citep{berner_tree_2017}. Warming temperatures have
tripled the frequency and quadrupled the size of wildfires since 2000
{[}\citep{iglesias2022}{]}.

Landscape-scale forest restoration efforts have been planned or
implemented across much of Arizona. For example, the Four Forest
Restoration Initiative (4FRI) includes plans for restoration across over
1 million hectares of Arizona's forests
\citep{schultz_collaborative_2012}. The primary goal of restoration
efforts is to reduce wildfire risk
\citep{allen_ecological_2002, friederici2013}. However, numerous studies
have linked forest treatments to increased water yields in semi-arid
forests and have emphasized the role of forest restoration in improving
hydrologic services and increasing water availability
\citep{bosch_review_1982, baker_effects_1986, gottfried_moderate_1991, smerdon_overview_2009, zou_streamflow_2010, wyatt_estimating_2013, moreno_modeling_2015, simonit_impact_2015, wyatt_semiarid_2015, odonnell_forest_2018, schenk_impacts_2020, hibbert1979}.

While the connection between forest treatment and water yield is well
documented, the response of forests to treatments is complex and
non-linear and differs across forest types, with treatment level, and
along aspect and elevational gradients
\citep{del_campo_global_2022, biederman_streamflow_2022, zou_streamflow_2010, hibbert1979, moore_physical_2005}.
Regardless of the potential for increased water yield the enhancement of
groundwater recharge rarely, if ever, ranks among the primary
motivations for forest treatment even among projects with the stated
goal of improving watershed health
\citep{stanturf2014, filoso2017, allen_ecological_2002, friederici2013, odonnell2016}.
This study examines forest restoration through the lens of groundwater
recharge enhancement and identifying potential recharge zones. All else
held equal, we map suitability for forest thinning with the goal of
enhancing recharge. Suitability maps like these may complement (or
supplement) existing frameworks for prioritizing landscape-scale forest
management.

Suitability mapping, and particularly GIS-based Multi-criteria decision
making (GIS-MCDM), is widely used to map potential recharge zones and
areas suitable for Managed Aquifer Recharge (MAR), but to our knowledge,
it has not yet been implemented to map recharge enhancement potential
through forest thinning \citep{fathi2021, rajashekar2023}. Pairwise
comparisons were made between criteria including average precipitation,
snow dominance, slope, aspect, landscape morphology, forest density,
lithology, and hydrologic soil type, and the Analytic Hierarchy Process
(AHP) was used as a weighting method.

\begin{center}\rule{0.5\linewidth}{0.5pt}\end{center}

Forest treatments such as thinning and burning can significantly impact
the hydrologic cycle of forests \citep{del_campo_global_2022}. For
example, forest thinning in Arizona has been associated with increased
snow cover days
\citep{sankey_multi-scale_2015, belmonte_uav-based_2021, donager_integrating_2021},
greater soil moisture \citep{belmonte_soil_2022, sankey_thinning_2022},
and greater forest canopy moisture \citep{sankey_regionalscale_2021}.
However, the response of forests to treatments is complex and non-linear
and differs across forest types, with treatment level, and along aspect
and elevational gradients
\citep{del_campo_global_2022, biederman_streamflow_2022, zou_streamflow_2010, hibbert1979, moore_physical_2005}.

Water yield can decrease with reductions in forest cover in drier
forests with little topographic shading or SW aspects due to increased
water use by remaining vegetation and increased snow sublimation or
direct evaporation of soil moisture
\citep{biederman_recent_2015, goeking_forests_2020}. Biederman and
others \citep{biederman_streamflow_2022} found that low-elevation
forests in Arizona may produce less streamflow following reductions in
canopy cover due to wildfire, highlighting the importance of elevation
and particularly water-energy asynchrony to water yield
\citep{webb2024}. The effects of forest treatment appear to have little
or no effect on water yield in areas receiving less than 500mm of annual
precipitation
\citep{biederman_streamflow_2022, carroll_evaluating_2016, adams_ecohydrological_2012, zou_streamflow_2010, hibbert1979}.

\subsection{Literature Reviewed}\label{literature-reviewed}

\subsection{Study Area}\label{study-area}

Coconino National forest is 812605 Ha

Ponderosa Pine covers 336,046 Ha of the Coconino = 41.35\%

\subsection{Acknowledgments}\label{acknowledgments}

Phasellus interdum tincidunt ex, a euismod massa pulvinar at. Ut
fringilla ut nisi nec volutpat. Morbi imperdiet congue tincidunt.
Vivamus eget rutrum purus. Etiam et pretium justo. Donec et egestas sem.
Donec molestie ex sit amet viverra egestas. Nullam justo nulla,
fringilla at iaculis in, posuere non mauris. Ut eget imperdiet elit.

\subsection{Open research}\label{open-research}

Phasellus interdum tincidunt ex, a euismod massa pulvinar at. Ut
fringilla ut nisi nec volutpat. Morbi imperdiet congue tincidunt.
Vivamus eget rutrum purus. Etiam et pretium justo. Donec et egestas sem.
Donec molestie ex sit amet viverra egestas. Nullam justo nulla,
fringilla at iaculis in, posuere non mauris. Ut eget imperdiet elit.

\subsection*{References}\label{references}
\addcontentsline{toc}{subsection}{References}

\renewcommand{\bibsection}{}
\bibliography{bibliography.bib}




\end{document}
